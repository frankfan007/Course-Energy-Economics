\documentclass[letterpaper,10pt]{article}
\usepackage{fullpage}
\usepackage{hyperref}
\newcommand{\Term}{Winter 2014}
\newcommand{\Office}{on Monday from 12:00-1:00}



\date{}
%opening
\title{Energy Economics\\ 
EC XXX, XXX\\
\Term}

\author{James Woods}

\begin{document}

\maketitle

\section{Introduction}

\section{Prerequisites}


\section{Contact Information}

I encourage you to post your questions on Piazza.  You can find our class page at: \url{http://piazza.com/pdx/xxx}.  There is also a link in D2L. Please get signed up as soon as possible. If you have any problems or feedback for the developers, email team@piazza.com.

If you have a question about course mechanics, interpretation of the
syllabus, or where something is located, there is a folder for
that called ``Logistics''. I will check piazza on a regular basis during regular business hours, and answer questions. I also expect you and your peers to help to each other and will indicate the good answers and make clarifications.  

If you send me an email with a question that should be asked in a piazza, I
will ask you to post it there rather than answer in email.


The TA assigned to the class is XXX. 

 Her contact information is:
 \begin{itemize}
 \item Office Hours: Office Hours: TBA, CH 230
 \item Email/gtalk: x@pdx.edu
%  \item D2L Chat on occasion.
 \end{itemize}
 

My office is in CH 241-O.  The best ways of contacting me, in
decreasing order of effectiveness:
\begin{itemize}
\item IM/hangout: woodsj@pdx.edu
\item Phone/text: (503) 465-4883
\item email:woodsj@pdx.edu (Answered only on M W and F)
\item D2L email: Do not use.  I will not respond.
\end{itemize}

You are encouraged to use the IM function of your PSU email account rather than email. This will allow me to get back to you more quickly and more conversationally. Please be aware that I that I will likely not reply till the next day if you IM me after 6pm or on the weekend.  

My office hours will be held \Office \ through the last week of class, not including finals week. There is no need to make an appointment for these hours -- just come. If you can't attend regular office hours, please check my calendar link \url{http://x/}. I will make a limited number of 15 minute slots available each week. 

\subsection{Use of D2L}

D2L will be essential to the class for:

\begin{itemize}
\item Access to your grades

\item Links to the library's electronic reserve. All required reading
  assignments are provided either electronically or on reserve.  There is no required
  packet or book.

\item Links to additional resources and the locations of students'
  public work.

\item A updated calendar of key dates and expected readings for each
  class meeting.
 
\end{itemize}
  


\section{Basis For Grade}

This class will be taught as a collaborative seminar with limited
traditional lecture. There are no exams but there will be a
considerable amount of writing and analysis, presentations by teams of
student researchers.

\subsection{Individual Assignments}
 Each student is required to provide one paragraph responses to
  each paper assigned in the class and a penetrating non-factual
  discussion prompt on the text.  The paragraph must submitted in the dropbox folder with the same title as the paper or chapter by the due date, generally the night before class. This will be evaluated by the instructor on a 0 to 2 scale. 

The following rubric will be used for response paragraphs.
\begin{description}
 \item[0 Points] Base
 \item[1 Points] Response deals with details of the paper or chapter beyond what can be gleaned from the abstract, introduction and conclusion.
 \item[2 Points] all required for a 1 and a discussion prompt used in class.
\end{description}

These individual assignments will constitute 20\% of your grade.

\subsection{Team Assignments}

Teams will be assigned by the instructor after the first week of
class.  Ideally, they will be composed of at least one graduate student and one
undergraduate student, but exceptional undergraduates may take leadership possitions in the team.  Assignments will be made based a survey of
completed math, economics and statistics courses.  The intent is to
ensure that each team has the skills it needs to succeed.

The teams are intended to encourage cooperative learning where
graduate students can assist undergraduate students with concepts they
are unsure of and those with exceptional skills can share insight with
others.

A team may remove a member by an anonymous 3/4th majority vote of a
current members. Removed members will act as a team of one until other
teams remove members or students join the class.  Removed members may
join other teams by an anonymous 3/4th majority vote of the receiving
team. All votes will be conducted by the instructor should two
students on a team request a vote. \emph{The instructor reserves the right to reassign students to teams should there be large differences in team size.}

\subsubsection{Small Team Assignments}

Teams will be assigned to explain one or more diagrams, tables,
subsequent literature or equations in the papers and evaluations we discuss. Teams are expected to create intuitive explanations, make critical statements,
and field reasonable questions. Expect these assignments at least once
a week. Performance will be evaluated by the instructor on a 0 to 5
scale.

\begin{description}
\item[0 Points] Base
 \item[1 Points] Rudimentary synopsis that simply repeats what was in the document.
 \item[2 Points] Synopsis that also includes an example and explanation that goes beyond what was in the document, or that explains in greater detail what was in the document.
 \item[3 Points] 2 AND able to respond to simple questions.
 \item[4 Points] 3 AND able to respond to questions about implications and the effects of changes in assumption or form.
 \item[5 Points] Exemplary example of 4. 
 \end{description}

Some of these assignments will be given ahead of time and require a presentation.  Others will be given the day of and constitute a discussion topic that will be memorialized in a shared document constructed in class.

These will be worth a total of 40\% of your final grade.

\subsubsection{Larger Team Assignments}

Each of the reading sections will have a larger project associated with them that will require somewhat more work.  These will be evaluated on a scale similar to the small assignments and constitute the remaining 40\% of your final grade.


\subsection{Reading}

Because of the large number of topics that could be included in this course only a fraction of this reading list will be completed.  Only the first section, the overview, is required.

\subsection{Required Introduction to Energy Economics}
  \begin{enumerate}
    \item Introduction and Background
  
  \end{enumerate}
  
When the overview is nearing completion the class will decide on the next topic via an exhaustive ballot.  Not all of the sections will be available immediately since some require prior completion of one or more topics.

\subsection{Topics}

\begin{enumerate}
  \item Energy Efficiency Policies
  \item Renewable Energy Policies


  \item Provisioning Ancilary services
  \item Electricity in Hawaii
\end{enumerate}

\end{document}
