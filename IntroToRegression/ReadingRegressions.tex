% Options for packages loaded elsewhere
\PassOptionsToPackage{unicode}{hyperref}
\PassOptionsToPackage{hyphens}{url}
%
\documentclass[
  ignorenonframetext,
]{beamer}
\usepackage{pgfpages}
\setbeamertemplate{caption}[numbered]
\setbeamertemplate{caption label separator}{: }
\setbeamercolor{caption name}{fg=normal text.fg}
\beamertemplatenavigationsymbolsempty
% Prevent slide breaks in the middle of a paragraph
\widowpenalties 1 10000
\raggedbottom
\setbeamertemplate{part page}{
  \centering
  \begin{beamercolorbox}[sep=16pt,center]{part title}
    \usebeamerfont{part title}\insertpart\par
  \end{beamercolorbox}
}
\setbeamertemplate{section page}{
  \centering
  \begin{beamercolorbox}[sep=12pt,center]{part title}
    \usebeamerfont{section title}\insertsection\par
  \end{beamercolorbox}
}
\setbeamertemplate{subsection page}{
  \centering
  \begin{beamercolorbox}[sep=8pt,center]{part title}
    \usebeamerfont{subsection title}\insertsubsection\par
  \end{beamercolorbox}
}
\AtBeginPart{
  \frame{\partpage}
}
\AtBeginSection{
  \ifbibliography
  \else
    \frame{\sectionpage}
  \fi
}
\AtBeginSubsection{
  \frame{\subsectionpage}
}
\usepackage{lmodern}
\usepackage{amssymb,amsmath}
\usepackage{ifxetex,ifluatex}
\ifnum 0\ifxetex 1\fi\ifluatex 1\fi=0 % if pdftex
  \usepackage[T1]{fontenc}
  \usepackage[utf8]{inputenc}
  \usepackage{textcomp} % provide euro and other symbols
\else % if luatex or xetex
  \usepackage{unicode-math}
  \defaultfontfeatures{Scale=MatchLowercase}
  \defaultfontfeatures[\rmfamily]{Ligatures=TeX,Scale=1}
\fi
% Use upquote if available, for straight quotes in verbatim environments
\IfFileExists{upquote.sty}{\usepackage{upquote}}{}
\IfFileExists{microtype.sty}{% use microtype if available
  \usepackage[]{microtype}
  \UseMicrotypeSet[protrusion]{basicmath} % disable protrusion for tt fonts
}{}
\makeatletter
\@ifundefined{KOMAClassName}{% if non-KOMA class
  \IfFileExists{parskip.sty}{%
    \usepackage{parskip}
  }{% else
    \setlength{\parindent}{0pt}
    \setlength{\parskip}{6pt plus 2pt minus 1pt}}
}{% if KOMA class
  \KOMAoptions{parskip=half}}
\makeatother
\usepackage{xcolor}
\IfFileExists{xurl.sty}{\usepackage{xurl}}{} % add URL line breaks if available
\IfFileExists{bookmark.sty}{\usepackage{bookmark}}{\usepackage{hyperref}}
\hypersetup{
  pdftitle={Reading Regression Output Without Taking Econometrics},
  pdfauthor={James Woods},
  hidelinks,
  pdfcreator={LaTeX via pandoc}}
\urlstyle{same} % disable monospaced font for URLs
\newif\ifbibliography
\usepackage{graphicx,grffile}
\makeatletter
\def\maxwidth{\ifdim\Gin@nat@width>\linewidth\linewidth\else\Gin@nat@width\fi}
\def\maxheight{\ifdim\Gin@nat@height>\textheight\textheight\else\Gin@nat@height\fi}
\makeatother
% Scale images if necessary, so that they will not overflow the page
% margins by default, and it is still possible to overwrite the defaults
% using explicit options in \includegraphics[width, height, ...]{}
\setkeys{Gin}{width=\maxwidth,height=\maxheight,keepaspectratio}
% Set default figure placement to htbp
\makeatletter
\def\fps@figure{htbp}
\makeatother
\setlength{\emergencystretch}{3em} % prevent overfull lines
\providecommand{\tightlist}{%
  \setlength{\itemsep}{0pt}\setlength{\parskip}{0pt}}
\setcounter{secnumdepth}{-\maxdimen} % remove section numbering

\title{Reading Regression Output Without Taking Econometrics}
\author{James Woods}
\date{}

\begin{document}
\frame{\titlepage}

\begin{frame}{Just Basics}
\protect\hypertarget{just-basics}{}

Doing Econometrics well is hard.

\begin{itemize}
\tightlist
\item
  EC 460 will teach you the basics of single-equation regression, e.g.,
  demand estimation, and how to fix some things.
\item
  Make you overconfident.
\item
  Functional work requires:

  \begin{itemize}
  \tightlist
  \item
    Systems equation estimation, e.g., supply and demand at the same
    time.
  \item
    Discrete choice, e.g., Yes/No or make and model of a car.
  \item
    Knowledge of how to get causality, experimental, e.g., RCT, and
    quasi-experimental methods, e.g., regression discontinuity.
  \end{itemize}
\end{itemize}

\end{frame}

\begin{frame}{BASICS Single Equation}
\protect\hypertarget{basics-single-equation}{}

Explain something, often called the left-hand side or endogenous
variable, with explanatory, right-hand side or exogenous variables.

\[Weight = \alpha + \beta Height + \epsilon\]

\begin{itemize}
\tightlist
\item
  Weight is left-hand side
\item
  \(\alpha\) intercept term, expected weight given you have no height.
\item
  \(\beta\) How much your weight increases for every inch of height.
\item
  \(\epsilon\) How far off we were.
\item
  All the greek letters are random variables. We estimate the means and
  variances of those.
\end{itemize}

Generating Fake Data with Weight = 20 + 2.75 Height + n(0,40)

\end{frame}

\begin{frame}{Fake Data}
\protect\hypertarget{fake-data}{}

\includegraphics{ReadingRegressions_files/figure-beamer/unnamed-chunk-1-1.pdf}

\end{frame}

\begin{frame}{Regression Output}
\protect\hypertarget{regression-output}{}

\begin{table}[!htbp] \centering 
  \caption{} 
  \label{} 
\begin{tabular}{@{\extracolsep{5pt}}lc} 
\\[-1.8ex]\hline 
\hline \\[-1.8ex] 
 & \multicolumn{1}{c}{\textit{Dependent variable:}} \\ 
\cline{2-2} 
\\[-1.8ex] & Weight \\ 
\hline \\[-1.8ex] 
 Height & 2.890$^{***}$ \\ 
  & (0.408) \\ 
  & \\ 
 Constant & 27.994 \\ 
  & (29.040) \\ 
  & \\ 
\hline \\[-1.8ex] 
Observations & 100 \\ 
R$^{2}$ & 0.338 \\ 
Adjusted R$^{2}$ & 0.332 \\ 
Residual Std. Error & 42.340 (df = 98) \\ 
F Statistic & 50.114$^{***}$ (df = 1; 98) \\ 
\hline 
\hline \\[-1.8ex] 
\textit{Note:}  & \multicolumn{1}{r}{$^{*}$p$<$0.1; $^{**}$p$<$0.05; $^{***}$p$<$0.01} \\ 
\end{tabular} 
\end{table}

\end{frame}

\begin{frame}{Parts}
\protect\hypertarget{parts}{}

Regression as a whole:

\begin{itemize}
\tightlist
\item
  \(R^2\) fraction of variation in LHS explained by variation in RHS.

  \begin{itemize}
  \tightlist
  \item
    Because if you add more variables \(R^2\) goes up, Adjusted \(R^2\),
    penalizes for having more variables.
  \end{itemize}
\item
  \(F\) Difference between model that is mean of LHS alone, just the
  intercept, vs model with RHS.
\end{itemize}

The parameters:

\begin{itemize}
\item
\end{itemize}

\end{frame}

\end{document}
